\documentclass[11p]{article}
% Packages
\usepackage{amsmath}
\usepackage{graphicx}
\usepackage{fancyheadings}
\usepackage[swedish]{babel}
\usepackage[
    backend=biber,
    style=authoryear-ibid,
    sorting=ynt
]{biblatex}
\usepackage[utf8]{inputenc}
\usepackage[T1]{fontenc}
\usepackage{titlesec}
\usepackage{hyperref}

%Källor
\addbibresource{references.bib}
\graphicspath{ {./images/} }

% Lite variabler
\def\email{Gabrielnilsson.hogb@elev.ga.ntig.se}
\def\foottitle{}
\def\name{Gabriel Nilsson Högberg}


\begin{document}
    \begin{titlepage}
        \centering

        \vspace*{1cm}

        % Title and subtitle are enclosed between two rules.
        \rule{\textwidth}{1pt}

        % Title
        \vspace{.7\baselineskip}
        {\huge \textbf{Inte tillgänglig för alla}}

        % Subtitle
        \vspace*{.5cm}
        {\LARGE En studie om kommunala webbplatsers tillgänglighet}

        \rule{\textwidth}{1pt}

        \vspace{1cm}

        % Set this size for the remaining titlepage.
        \large

        % Authors side by side, using two minipages as a trick.
        \begin{minipage}{.5\textwidth}
            \centering
            Gabriel Nilsson Högberg \\
            {\normalsize \url{Gabrielnilsson.hogb@elev.ga.ntig.se}}
        \end{minipage}%

        \vspace{3cm}

        % Report logo.
        \includegraphics[width=.7\textwidth]{../images/nti_logo.png}

        \vfill

        % University and date information at the bottom of the titlepage.
        NTI Gymnasiet Umeå \\
        Teknikprogrammet\\
        Gymnasiearbete\\
        Datum: \today \\
        Handledare: Jens Andreasson
    \end{titlepage}

    \begin{center}
        \large
        \vspace{0.9cm}
        \textbf{Abstract}
    \end{center}
    In Sweden, around 95 percent used the internet in 2022 \textcite{SCB} and the fact that so many people used the internet placed ever greater demands.
    This means that the websites must work and ensure that everyone can use them in a convenient way.
    The purpose of the analysis was to investigate how well communal websites followed accessibility guidelines.
    This was especially important for public websites because there are laws and requirements for accessibility when it comes to the web.

    A manual test was done because there are no good tools that check the WCAG 2.1 criteria.
    WAVE was used to streamline manual analysis with contrasts, ARIA references and alt tags.
    Finally, Lighthouse was used to find issues that cannot be found by evaluation tools or the manual check.

    The most common criteria that the web pages did not reach was 1.1.1, 2.5.3 and 2.4.6.
    These criterias explained that alternative text must be present on images, input elements such as links or buttons need to have an alternative text that matches the text on the button or link and that header and labels for forms and interactive controls are informative.

    Most of the requirements were met except for a few on all web pages.
    Not all web pages passed criteria 1.1.1 within WCAG 2.0.
    Alternative text can be a simple thing that many page developers may miss or not fully meet.
    Although it was a worrying trend that the web pages were not able to have the same alternative text that is displayed visually with, for example, links and buttons according to criteria 2.5.3.

    Few problems were found with the web pages and the problems found should be improved to get more accessible pages.
    The DOS law contains requirements to be achieved and the Authority for Digital Management recommended following the European standard EN 301 549 which referred to WCAG 2.1 at AA level.
    Improvements that could be made for a similar study could be to make use of the new WCAG 2.2 or WCAG 3.0 which is being developed to become the new web standard.
    A deeper investigation on the various pages of the web pages could also make for a more detailed work.

    \newpage
    \tableofcontents

% i Sverige har vi normalt inget indrag vid nytt stycke
    \setlength{\parindent}{0pt}
% men däremot lite mellanrum
    \setlength{\parskip}{10pt}

    
    \section{Inledning}
    Webben som vi känner till idag har blivit en av världens mest använda verktyg sedan det släpptes för allmänheten 1993.
    I världen är det runt 4,39 miljarder personer som använder sig av internet vilket är över hälften av den globala populationen \textcite{history_WWW}.
    I Sverige är det runt 95 procent som använder sig av internet \textcite{SCB} och att så många personer använder internet ställer allt större krav.
    Detta innebär att webbplatserna ska fungera och att alla kan använda dem på ett bekvämt sätt.
    Det gör även att det ställs ett större krav på att webbsidorna ska vara tillgängliga.
    Ett stort ansvar hamnar då på de som utvecklar kommunala webbsidor att webbplatsen ska vara tillgänglig oavsett vilken enhet man använder.
    Ett extremt stort spektrum av behov behöver täckas för att kalla en webbsida tillgänglig.

    Den svenska myndigheten för digital förvaltning (DIGG) ger denna anledning till varför fokuset på tillgänglighet är viktigt:

    ``Omkring 16 procent av den globala befolkningen uppskattas att ha en signifikant funktionsnedsättning och denna siffra fortsätter att öka bland annat med åldrande befolkning.
    För personer med funktionsnedsättning kan god tillgänglighet vara helt avgörande för att kunna ta del av information eller utföra ett ärende på en offentlig webbplats \textcite{Digg_tillganglighet}.
    ``

    World Wide Web Consortium (W3C) arbetar med att förbättra Web content accessibility guidelines (WCAG) för att ge en standard om tillgänglighet på webben och för webbutvecklare.
    Flera webbutvecklare har valt att använda sig av denna standard för att skapa en sida från ett tillgänglighets perspektiv.
    Jag anser att det skulle vara intressant att undersöka svenska kommunala webbsidor och ifall dem följer tillgänglighetsstandarderna WCAG 2.0 och 2.1 och jämföra dem med andra kommuners webbsidor.
    
    \section{Syfte och Frågeställning}
    Undersökningen syftar till att ta reda på hur sidorna möter kriterier samtidigt som prestanda, tillgänglighet och bästa praxis kommer att mätas med hjälp av ett antal automatiserade tester.

    Syftet för undersökningen är att ta reda på hur väl kommunala webbplatser följer tillgänglighetsriktlinjer.
    Detta är viktigt för att så många som möjligt ska kunna använda internet på ett tillräckligt bra sätt utan svårigheter och för de offentliga webbsidorna eftersom det är lag och krav på tillgänglighet när det kommer till webben.

    I denna undersökning kommer följande kommunala sidor att undersökas:

    \begin{itemize}
        \item eslov.se
        \item goteborg.se
        \item helsingborg.se
        \item huddinge.se
    \end{itemize}

    Utifrån dessa frågeställningar:

    \begin{itemize}
        \item Möter webbsidan tillgänglighetskriterier från WCAG 2.0 och 2.1?
        \item Vilka enskilda kriterier misslyckades webbsidorna mest med?
        \item Vilken prestanda, tillgänglighet har sidorna och följer de god praxis.
    \end{itemize}
    
    \section{Bakgrund}
    
    \subsection{Vad är webbtillgänglighet?}
    Tillgänglighets aspekter kan skilja från person till person men vad exakt tillgänglighet är kan vara svårt att sätta fingret på.
    För denna studie används Steve Krug och hans egna definition av webbtillgänglighet i sin bok Don't Make Me Think revisited: A Common Sense Approach to Web Usability \textcite{Krug}.
    Steve Krug är ett UX-proffs som är känd för sin bok ``Don't Make Me Think`` som handlar om interaktionen mellan människa, dator och användarvänligheten på webben \textcite{Krug_Wikipedia}.

    Det är inte bara Steve Krug som tänker på detta sätt.
    Jakob Nielsen som är konsult inom dator- och webbanvändbarhet anses vara en av de ledande auktoriteterna inom användbarhet och tänker på ett likvärdigt sätt som Steve Krug \textcite{Jakob_Nielsen}.
    Han är känd för att ha kritik mot populära webbplatser och mot överdriven användning utav grafik och animeringar som inte hjälper användbarheten.
    Nielsens tio heuritiska principer liknar de punkter som Krug har gjort för sin egna bok.

    Steve Krug och den definitionen av användbarhet är följande:

    ``Du kommer hitta olika definitioner av användbarhet men ofta bryts det ned i olika punkter som;``

    \begin{itemize}
        \item Användbar: Gör den något som användaren behöver?
        \item Lärbar: Kan användaren lära sig hur man använder detta?
        \item Minnesvärd: Behöver användaren lära sig om denna sak varje gång den används?
        \item Effektiv: Får den jobbet gjort på en rimlig tid med den insats som denna användare gör?
        \item Önskvärd: Vill användaren ha detta?
        \item Trevlig: Tycker användaren att detta är trevligt eller till och med roligt att använda?
    \end{itemize}

    När något är användbart ska alltså en person som har erfarenheter och förmågor kunna lista ut hur man använder en produkt utan att det ska ta för mycket tid och energi än vad som egentligen behövs.

    \subsection{Vad är Webb Accessibility Initiative? (WAI)}
    WAI är ett initiativ från W3C som är till för att öka prioriteten för användbarhet för dem med funktionsnedsättningar.
    Initiativet utvecklas genom att arbeta med olika industrier, organisationer, regeringar och mer för att runt om i världen göra webben mer tillgänglig.
    WAI har några primära aktiviteter som \textcite{WAI}:

    \begin{itemize}
        \item Garantera att W3C standarder utvecklas med tankar för tillgänglighet
        \item Utveckla tillgänglighets riktlinjer för webben och appar
        \item Utveckla resurser för att förbättra webbtillgänglighets utvärderingar
        \item Fortsätta lära ut om webbtillgänglighet
        \item Samarbeta med undersökningar och andra utvecklingar som kan påverka webbtillgänglighet i framtiden
        \item Påverka fler att tänka på webbtillgänglighetsstandarder
    \end{itemize}

    \subsection{HTML och element}
    HTML är ett märkspråk som webbsidor använder sig utav och innebär att taggar och element förklarar hur webbsidan ska presenteras för användaren.
    Ett element i sig skulle kunna vara exempelvis en bild, en knapp eller en tabell och fungerar som byggstenar för hela webbsidan \parencite{codeBean}.

    \subsection{Alt-Taggar och WAI-ARIA}
    En alt-tagg är en alternativ text till img element på en webbsida och ska förklara och ge kontext till vad den är kopplad till.
    Ett bra exempel för detta är text som kan berätta vad som är på bilden och ge kontext varför den är här på sidan \textcite{Nordström}.
    WAI-ARIA används för att göra webbinnehåll mer tillgängligt för personer som använder till exempel hjälpmedel som skärmläsare och har blivit en teknisk standard. \textcite{ARIA}
    Problemet med WAI-ARIA är att det oftast används felaktigt och gör en webbsida mer otillgänglig istället eftersom personer med hjälpmedel ska kunna ta en del av webbplatser används ARIA-attribut som stöd i HTML-kod.
    Med korrekt användning av WAI-ARIA kan en webbläsare översätta HTML-kod till en skärmläsare som i sin tur läser upp för användaren om till exempel en meny är öppen eller stängd \textcite{7minds}.


    \subsection{Kontraster}
    Ett av de vanligare felen som kan uppstå på en webbsida är dåliga kontraster.
    Olika kontrastfel uppstår då det är svårt att se innehållet på en webbsida när färgerna liknar varandra.
    Detta kan till exempel vara en text som inte syns på grund av att bakgrundsfärgen har för dålig konstrast jämfört med själva texten.
    Det gör att ett stort problem uppstår speciellt för de som har synnedsättning eller nedsatt färgseende \textcite{Digg_2021}.
    Exempel på kontrastfel syns nedan i figur 1.

    \begin{figure}[hbt!]
        \includegraphics[width=0.8\textwidth]{../images/KontrastExempel.jpg}
        \caption{ Kontrastfel hittade med hjälp av WAVE }
    \end{figure}

    \section{Vad är World Wide Web Consortium? (W3C)}
    World Wide Web Consortium även förkortat till W3C skapades av Tim Berners-Lee och arbetar med att skapa en webb som är tillgänglig för alla.
    W3C skapar och utvecklar olika webbstandarder för att uppnå detta.
    W3Cs egna dokument om Web Content Accessibility Guidelines har blivit standarden inom webbutvecklings industrin \textcite{W3C}.

    \subsection{Vad är Web Content Accessibility Guidelines? (WCAG)}
    Web Content Accessibility Guidelines är ett dokument gjort av W3C som har riktlinjer på hur vi ska göra webben tillgänglig för så många som möjligt.
    W3C beskriver själv vilka områden som WCAG ska behandla:

    ``Genom att följa dessa riktlinjer kommer innehållet att bli mer tillgängligt för en större andel av personer med funktionsnedsättning, inklusive hjälpmedel för blindhet och nedsatt syn, dövhet och hörselnedsättning, begränsad rörelse, talsvårigheter, ljuskänslighet och kombinationer av dessa, och vissa hjälpmedel för inlärningssvårigheter och kognitiva begränsningar; men kommer inte att tillgodose alla användarbehov för personer med dessa funktionshinder.`` \textcite{WCAG}

    WCAG har även olika krav som indelas i möjlighet att uppfatta, hanterbarhet, begriplighet och hur robust en sida är.
    Möjlighet att uppfatta innehåller att en person ska kunna uppfatta sidan och att innehållet ska kunna presenteras på olika sätt som t.ex text kan konverteras till större stil.
    Hanterbarhet innebär att sidan ska kunna navigeras utan mus, alltså med tangentbord eller andra hjälpmedel.
    Begripligheten för sidan ska vara att det finns inmatningsstöd som hjälper användaren undvika eller rätta till misstag.
    Till sist ska sidan vara robust, innehållet ska vara robust för ett brett spektrum av användarprogram inklusive hjälpmedel som skärmläsare.
    Alla dessa krav tillsammans gör att en sida kan vara så användarvänlig och tillgänglig för dem flesta personerna. \textcite{Digg_2023}

    \section{Testverktyg}

    \begin{itemize}
        \item AChecker
        \item WAVE
        \item Lighthouse
    \end{itemize}

    \subsection{AChecker}
    AChecker är ett tillgänglighetsverktyg som Web Accessibility Initiative (WAI) beskriver som.

    "Interaktiv, internationell och justerbar webbtillgänglighetsgranskare.
    AChecker tillåter användare att göra sina egna riktlinjer och bearbeta sina egna tillgänglighets granskningar.
    Detta är baserat på Open Accessibility Checks (OAC)" \textcite{AChecker}

    AChecker använder olika riktlinjer och standarder för att kontrollera att en webbsida möter dessa standarder och riktlinjer som testet använder sig av.
    Efter att verktyget granskat webbsidan presenteras resultat i tre olika nivåer.

    \begin{itemize}
        \item Kända problem: Problem som programmet hittar och vet är fel på webbsidan
        \item Sannolika problem: Problem som kan vara fel men behövs kollas manuellt
        \item Potentiella problem: Problem som programmet inte själv kan förstå eller hantera och måste också manuellt kollas ifall det verkligen är ett fel.
    \end{itemize}
    
    \subsection{Web Accessibility Evaluation Tools (WAVE)}
    Web Accessibility Evaluation Tools (WAVE) är ett tillgänglighetsverktyg som går att lägga till i chrome som en extension.
    WAVE skapades 2001 av Webb Accessibility In Mind (webAIM) i Utah State University \textcite{WAVE}.
    Programmet i sig använder sig av WCAG som riktlinjes standard och som webbsidorna testas på.
    WAVE är även uppskriven i WAI tillgänglighetsprogram listan som rekommenderas att använda.

    \subsection{Lighthouse}
    Lighthouse är gjort av Google och är ett verktyg till för att undersöka webbsidor i fem olika kategorier och sedan lägger ett betyg på en skala mellan 0 - 100.
    Det fem olika kategorierna som undersöks av lighthouse är: prestanda, tillgänglighet, bästa praxis, sökmotoroptimering (SEO) och progressiv webbapp.
    Man får även kommentarer och sedan säger den till om vad som är fel med någon av dessa fem olika kategorier.
    Lighthouse indikerar även hur väl man hamnar när man söker i google search.
    Ifall tillgänglighet delen på ens webbsida inte är särskilt bra påverkar det när personer söker efter denna sida vilket kan leda till högre användning.
    
    \section{Sveriges arbete med tillgänglighet}
    Sveriges arbete med tillgänglighet på webben baseras på de lagar som Sverige använder sig av.
    För att kunna uppnå kraven som DOS-lagen innehåller rekommenderar DIGG att följa den europeiska standarden EN 301 549 som hänvisar till WCAG 2.1 på AA nivå \textcite{Digg_Dos}.
    Däremot rekommenderas det att en webbsida ska uppnå kraven till AAA nivå.
    EN 301 549 innehåller alltså fler kriterier som ska följas av offentliga webbsidor.

    \section{Metod}
    Tillgänglighetsstandarden idag har utökats enormt och förbättrar användarupplevelsen för alla.
    Inom Sverige finns WCAG som riktlinjer och webbsidor bör följa och även uppnå de krav som riktlinjerna ger \textcite{Digg}.
    Det är för allas bästa att dessa används för att förbättra sina och andras webbsidor så att alla kan använda dem.

    \begin{itemize}
        \item Med hjälp av programmet AChecker kontrollerades en webbsida på WCAG 2.0 kraven och alla tre nivåer A, AA och AAA. Dessa nivåer angav ambitionsnivån på sidan och grunden för tillgänglighetskrav är AA.
        \item Ett manuellt test gjordes för att utöka kraven till WCAG 2.1, alltså krav som inte står i WCAG 2.0 med hjälp av WAVE.
        \item Det manuella testet undersökte ifall ARIA-referenser fungerar och om det fanns alt-taggar på bilder, loggor och mer.
        \item Lighthouse användes för att kolla sidornas prestanda, tillgänglighet och praxis.
    \end{itemize}

    För att undvika mänskliga misstag under undersökningen användes utvärderingsverktyg.
    Det manuella testet gjordes eftersom det inte finns tillräckligt bra verktyg som kontrollerade WCAG 2.1 kriterierna.
    WAVE användes för att effektivisera den manuella undersökningen med kontraster, ARIA-referenser och alt-taggar.
    Till sist användes Lighthouse för att hitta problem som inte kunde hittas av utvärderingsverktyg eller den manuella kontrollen.

    \subsection{Val av sidor}
    De sidor som valdes anses vara fyra av de bästa kommunerna som finns på nätet enligt ComputerSweden under 2020.
    Utifrån det tankesättet är det viktigt att se ifall sidorna fortfarande håller upp till tillgänglighetsstandarderna.

    Sidorna som valdes för denna undersökning är:

    \begin{itemize}
        \item eslov.se
        \item goteborg.se
        \item helsingborg.se
        \item huddinge.se
    \end{itemize}

    \section{Testverktygens användning}

    \subsection{AChecker}
    AChecker är en rekommendation från \textcite{AChecker} och används som en av tillgänglighets programmen för webbutvecklare.
    Programmet underlättar undersökningen av kraven inom WCAG 2.0 på alla tre nivåer A, AA och AAA.

    \subsection{Lighthouse}
    Sidorna som testas kommer köras 3 gånger med lighthouse för att få så bra data som möjligt på webbsidorna.
    Appar och annat kommer stängas av för att inte påverka och ifall något program är igång kommer det att skrivas ned.
    Sidorna körs två gånger för att kontrollera om det uppstår någon skillnad i resultaten.

    \subsection{Det manuella testet}
    Det manuella testet kommer att undersöka WCAG 2.1 eftersom programmen använt hittills inte täcker de kraven som står där.
    Därför kommer WAVE användas för att kolla ARIA-referenser, kontraster och sidornas kodstruktur.

    \subsection{Alt-taggar}
    För att undersöka alt-taggar kommer DevTools (inspektor) användas för att se till att html koden innehåller alternativ text till bilder och att dem stämmer men också knappar med namn och att deras namn också stämmer.

    \subsection{Analysering}

    Allt data som insamlas kommer att jämföras och analyseras med riktlinjerna i både WCAG 2.0 och 2.1.
    Detta görs för att se om webbsidorna följer dessa riktlinjer eller inte.

    \section{Resultat}
    
    \subsection{WCAG 2.0 och 2.1}
    Det mest vanliga kriteriet som webbsidorna inte når upp till är 1.1.1.
    Kriteriet förklarar att alternativ text ska finnas på bilder.
    Det andra kriteriet som inte häller webbsidorna når upp till är 2.5.3. Kriteriet innebär att alla inmatningselement som t.ex länkar eller knappar som visar text eller bild med text ska ha en alternativ etikett med likadant namn.
    Detta gör att det blir enklare att använda röstinmatning och hitta element som sedan kan läsas upp för att navigera webbsidor.
    Till sist är det kriterium 2.4.6 som innebär att rubriker och etiketter för formulär och interaktiva kontroller ska vara informativa.
    Antalet gånger som webbsidorna misslyckades med dessa kriterier syns i figur 2.

    \begin{figure}[hbt!]
        \includegraphics[width=0.8\textwidth]{../images/antalmisslyckande.jpg}
        \caption{ Antal misslyckande på kriterier 1.1.1, 2.4.6 och 2.5.3 }
    \end{figure}

    \subsection{Analysering}
    
    \subsubsection{Kriteriet 1.1.1, Alternativ text}
    Kriteriet 1.1.1 hanterar allt innehåll som inte är text och om det finns på sidan ska den ha ett textalternativ.
    Innehållet som kategoriseras i kriteriet 1.1.1 är följande:

    \begin{itemize}
        \item Bilder som är aktiva på sidan
        \item Interaktiva bilder
        \item Informativa bilder
        \item CSS bilder
        \item Dekorativa bilder
        \item Komplexa grafer och diagram
        \item Kartor
        \item Inmatningskontroll bilder (t.ex en knapp som ser ut som en bild istället för att ha text)
        \item CAPTCHA (Completely Automated Public Turing test to tell Computers and Humans Apart)
        \item Ljud och video innehåll
    \end{itemize}

    Efter analyseringen av webbsidornas resultat så visade det sig att alla webbsidor inte klarade av detta kriteriet eftersom det var minst ett misslyckande per sida.
    Bilderna som orsakade att detta kriteriet inte uppnåddes hade inte fullständiga alternativa texter eller text runt om som kunde förklara bilderna.
    Resultatet efter denna analys syns i figur 3.
    I figurerna 4 och 5 är det exempel på några av webbsidorna där det syns vad exakt som kan misslyckas.

    \begin{figure}[hbt!]
        \includegraphics[width=0.8\textwidth]{../images/resultat111.jpg}
        \caption{ Resultat av kriteriet 1.1.1 }
    \end{figure}

    \begin{figure}[hbt!]
        \includegraphics[width=0.8\textwidth]{../images/Eslov111.jpg}
        \caption{ Eslov.se med en bild som inte har ett alternativt namn }
    \end{figure}

    \begin{figure}[hbt!]
        \includegraphics[width=0.8\textwidth]{../images/Goteborg111.jpg}
        \caption{ Goteborg.se med en bild som inte har ett alternativt namn }
    \end{figure}

    \subsubsection{Kriteriet 2.4.6, Rubriker och etiketter}
    Kriteriet 2.4.6 handlar om rubriker eller etiketter på en webbsida och om detta finns ska dem vara informativa eftersom detta hjälper de som använder sig utav skärmläsare.
    Detta kriteriet innehåller inte att man ska använda sig utav rubriker och etiketter utan bara om de används ska de användas på rätt sätt, skrivna tydligt och meningsfullt.
    Däremot är det mer eller mindre ett krav för användbarheten.

    Eslov.se och huddinge.se har en rubrik som inte har någon text vilket gör att den inte är tydlig eller meningsfull.
    Helsingborg.se har en etikett till ett formulär som saknar en meningsfull text som gör att skärmläsare inte kan visa denna information korrekt.
    Rubrikerna och etiketterna som orsakade att detta kriteriet inte uppnåddes hade icke fullständiga rubriker och etiketter som inte var tillräckligt tydliga eller meningsfulla.
    Resultatet efter denna analys syns i figur 5.
    I figurerna 6, 7 och 8 synds det exempel från några webbsidor vad som misslyckades.

    \begin{figure}[hbt!]
        \includegraphics[width=0.8\textwidth]{../images/resultat246.jpg}
        \caption{ Resultat av kriteriet 2.4.6 }
    \end{figure}

    \begin{figure}[hbt!]
        \includegraphics[width=0.8\textwidth]{../images/Eslov246.jpg}
        \caption{ Eslov.se med en rubrik som inte innehåller något }
    \end{figure}

    \begin{figure}[hbt!]
        \includegraphics[width=0.8\textwidth]{../images/Huddinge246.jpg}
        \caption{ Huddinge.se med en rubrik som inte innehåller något }
    \end{figure}

    \begin{figure}[hbt!]
        \includegraphics[width=0.8\textwidth]{../images/Helsingborg246.jpg}
        \caption{ Helsingborg.se med ett formulär utan etikett }
    \end{figure}

    \subsubsection{Kriteriet 2.5.3, Inmatningselement}
    Kriteriet 2.5.3, hanterar innehåll som interaktiva element som till exempel knappar.
    Dessa element ska ha en alternativ etikett eller så måste etiketten ha samma namn som texten på elementet, detta är för att röstinmatning ska kunna hitta elementen.
    Om etiketten inte är detsamma som texten blir det svårt att använda röstinmatning för att navigera webbsidor.

    Resultatet efter denna analys syns i figur 8.
    Exempel kommer att visas från varje sida som inte uppfyllde kriteriet och sedan kommer det förklaras varför i figurens bildtext.
    eslov.se, helsingborg.se och goteborg.se hade felet att namnet på knapparna är fel som visas i figurer 9, 10 och 11

    \begin{figure}[hbt!]
        \includegraphics[width=0.8\textwidth]{../images/resultat253.jpg}
        \caption{ Resultat av kriteriet 2.5.3 }
    \end{figure}

    \begin{figure}[hbt!]
        \includegraphics[width=0.8\textwidth]{../images/Eslov253.jpg}
        \caption{ Eslov.se med en knapp som inte har ett alternativt namn }
    \end{figure}

    \begin{figure}[hbt!]
        \includegraphics[width=0.8\textwidth]{../images/Goteborg253.jpg}
        \caption{ Goteborg.se med en knapp som inte har ett alternativt namn }
    \end{figure}

    \begin{figure}[hbt!]
        \includegraphics[width=0.8\textwidth]{../images/Helsingborg253.jpg}
        \caption{ Helsingborg.se med en knapp som inte har ett alternativt namn }
    \end{figure}
<
    \subsection{Lighthouse}

    Resultatet från Lighthouse testerna kommer att visa sidornas prestanda, tillgänglighet och hur de arbetar utifrån bästa praxis.
    Resultaten från undersökningen visade att alla webbsidor hade högt medelvärde i prestanda och praxis.
    Däremot så hade goteborg.se och helsingborg.se betydligt lägre tillgänglighet ifall man jämför med resten av sidorna och detta syns i figur 12

    \begin{figure}[hbt!]
        \includegraphics[width=0.8\textwidth]{../images/Medelvarde.jpg}
        \caption{ Medelvärdet på lighthouse testerna utfört på dator versionen }
    \end{figure}

    Webbsidorna med lägst medelvärde på tillgänglighet var goteborg.se och helsingborg.se som endast hade 74 respektive 77 poäng utav 100.
    Utöver det uppnådde de flesta webbsidorna över 80 poäng om inte mer på prestanda och praxis.
    Webbsidan med bäst lighthouse resultat på datorn är eslov.se.

    \section{Diskussion}
    Baserat på undersökningens frågeställning och det resultat som undersökningen framställde så vill jag diskutera delar av resultatet.
    Utifrån frågan om sidorna kunde uppfylla tillgänglighetskraven WCAG 2.1, var de flesta kraven uppfyllda utom några enstaka på alla webbsidor.
    Det var få krav som inte uppfylldes vilket var positivt överraskande.
    Alla webbsidor klarade inte av kriteriet 1.1.1 inom WCAG 2.0 vilket syns i figur 3.
    Alternativ text kan vara en enkel sak som många utvecklare av sidor kan missa eller inte uppfylls helt.
    Det finns program som kan användas för att förenkla denna process med att skriva alt-taggarna.
    Däremot är det också viktigt att nämna att det inte behöver vara utvecklarna av sidan som orsakade denna miss.
    Det kan vara den person som hanterar text eller innehåll på sidan som inte gav utvecklaren den alternativa texten som behövs.
    Webbsidor kan också använda sig utav att gömma denna typ av innehåll när skärmläsare skulle användas, dock så går det att argumentera om detta är en tillräckligt bra lösning på problemet.
    Det innebär alltså att det kan vara ett designval eller liknande som man utvecklarna själva väljer att göra.

    Det är en oroväckande trend att webbsidorna inte klarar av att ha samma alternativt text som visas visuellt med till exempel länkar och knappar enligt kriteriet 2.5.3.
    Kriteriet kräver att alla inmatningselement med text eller bild ska ha samma namn.
    Detta görs för att förenkla användningen av röstinmatning.
    Alla sidor förutom huddinge.se hade en del på sidan som inte klarade av detta kriteriet på olika sätt.
    Detta går även för kriteriet 1.1.1 och alla webbsidor som jag undersökt med alternativ text istället och detta väcker frågan om hur mycket detta faktiskt prioriteras.
    
    \section{Slutsatser}
    I det stora hela hittades få problem med webbsidorna och det var få kriterier som inte uppfylldes vilket var positivt.
    Däremot bör de problem som hittades förbättras för att få mer tillgängliga sidor eftersom det finns faktiskt lagar webbsidor bör nå upp till.
    DOS-lagen innehåller krav som ska uppnås och Myndigheten för digital förvaltning rekommenderar att följa den europeiska standarden EN 301 549 som hänvisar till WCAG 2.1 på AA nivå.
    Förbättringar som skulle kunna göras för en liknande studie skulle kunna vara att använda sig utav nya WCAG 2.2 eller WCAG 3.0 som håller på att utvecklas för att bli den nya webbstandarden.
    En djupare undersökning på webbsidornas olika sidor skulle även göra ett mer utförligt arbete.

    \section{Referenser}
    \printbibliography[heading=none]

\end{document}